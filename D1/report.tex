\documentclass[11pt,a4paper]{ctexart}
\input{macros.tex}
\graphicspath{{figure/}}
\usepackage{enumerate}
%\usepackage{fancyhdr}
%\cpic{<尺寸>}{<文件名>}}用于生成居中的图片。
\newcommand{\cpic}[2]{
\begin{center}
\includegraphics[scale=#1]{#2}
\end{center}
}
%\cpicn{<尺寸>}{<文件名>}{<注释>}用于生成居中且带有注释的图片,其label为图片名。
\newcommand{\cpicn}[3]
{
\begin{figure}[H]
\cpic{#1}{#2}
\caption{#3\label{#2}}
\end{figure}
}

\crefname{equation}{}{}
\crefname{figure}{图}{图}
\crefname{footnote}{注释}{注释}
\crefname{table}{表}{表}
\title{实验D1 锁相放大器与弱信号测量(必做)}
\begin{document}
\maketitle
\begin{tabular}{|p{8em}|p{8em}|p{8em}|p{5em}|}
\hline
		实验方案 &实验记录  &分析讨论 &总成绩\\
		\hline
		        &          &          &  \\
	    \hline
	\hline 
	年级、专业: &17级物理学 &组号:& 6 \\
	\hline
	姓名:& 徐昊霆 &学号:&17353071  \\
	\hline
	日期:& \today &教师签名: &  \\
    \hline	
        \end{tabular}

\begin{enumerate}
 \item 实验报告由三部分组成:
 \begin{enumerate}
  \item[1)]预习报告:(提前一周)认真研读\textbf{\uline{实验讲义}},弄清实验原理;实验所需的仪器设备、用具及其使用(强烈建议到实验室预习),完成讲义中的预习思考题;了解实验需要测量的物理量,并根据要求提前准备实验记录表格(由学生自己在实验前设计好,可以打印)。预习成绩低于50\%者不能做实验{\color{red} (实验D2和D3需要提前一周的周四完成预习报告交任课老师批改,批改通过后,才允许做实验)}。

  \item[2)]实验记录:认真、客观记录实验条件、实验过程中的现象以及数据。实验记录请用珠笔或者钢笔书写并签名({\color{red}用铅笔记录的被认为无效})。{\color{red}保持原始记录,包括写错删除部分,如因误记需要修改记录,必须按规范修改。}(不得输入电脑打印,但可扫描手记后打印扫描件);离开前请实验教师检查记录并签名。
  \item[3)]分析讨论:处理实验原始数据(学习仪器使用类型的实验除外),对数据的可靠性和合理性进行分析;按规范呈现数据和结果(图、表),包括数据、图表按顺序编号及其引用;分析物理现象(含回答实验思考题,写出问题思考过程,必要时按规范引用数据);最后得出结论。
 \end{enumerate}
 \textbf{实验报告}就是预习报告、实验记录、和数据处理与分析合起来,加上本页封面。
 \item 每次完成实验后的一周内交\textbf{实验报告}。
 \item 除实验记录外,实验报告其他部分建议双面打印。
\end{enumerate}
\newpage
\tableofcontents
\newpage
\section{实验原理与方案}
\subsection{实验目的}
\begin{enumerate}
 \item[1.]了解锁相放大器工作原理和特点,掌握锁相放大器基本参数含义及锁相放大器的基本 操作;复习示波器的使用;(实验内容 1)
  \item[2.]掌握用锁相放大器检测出湮没于噪声中的弱信号方法;(实验内容 2)
  \item[3.]通过本实验更加直观的了解方波的高频成分,并掌握用锁相放大器检测微弱信号中的谐波;(实验内容 3)
\item[4.]掌握交流四引线法精密测量小阻抗的原理和方法,理解相位的物理含义及电压与电流之间相位差的测量方法;(实验内容 4,探索)
\item[5.] 学习 VISA 接口协议,通过 LabVIEW 环境用 PC 机控制锁相放大器数据采集;(实验内容 4)
\item[6.]了解 PN 结结电容,探究变容二极管内 PN 结电容与反偏电压的关系;(实验内容 5)
\item[7.] 通过实验了解热噪声的特征,理解热噪声的本质及其统计意义。(实验内容 6,探索)
\end{enumerate}


\subsection{仪器用具}
\begin{table}[H]
\cpic{0.4}{t1}
\caption{弱信号检测实验仪器用具}
\end{table}

\subsection{实验安全注意事项}

1. OE1022 锁相放大器输入端不能接入强信号;接入信号需限定在 1V 以下。

2. OE4004 稳流源的电流输出不能直接接入 OE1022 锁相放大器的输入端。

\subsection{实验原理}
\subsubsection{噪声的定义及数学描述}
现代测量中,所有的物理最终都转换为电压或光强进行记录和处理,我们称之为携带被测量物理量信息的电或光信号;然而所有的测量,即使完全由机器自动进执行,结果都不可避免地引入待测量以外的其他信息,如环境干扰、不确定等,在所测量的信号不可避免地携带了噪声。测量结果(信号$x(t)$)可以视作被测量信号$s(t)$与噪声$n(t)$的叠加$x(t) = s(t) + n(t) $。
\par
一般来说,信号是运载信息的工具,是信息的载体。任何携带某个现象属性或行为信息的物理量都可以作为信号,如光信号、声信号和电信号等。物理世界中的信号都是有规律地随时间或空间变化的。数学上以函数的形式$x(t)$描述信号。
\par
最简单的信号用单一频率的三角函数的描述
\beq \label{eq1}
s_\omega (t) = a \cos (\omega_0 t + \varphi) 
\eeq
它包含了三个最基本的信息,即振幅$a$、频率$\omega_0$和相位$\varphi$,它对于信号载体对应着不同的物理含义。在实际应用中,人们往往只用到其中的一个或两个基本信息来进行信号传递或处理,而对于科学探索可能会全方位地用到所有的信息。\cref{eq1}是在时域下对信号的描述;在频域下该信号的数学描述为:
\beq
s_e(\omega) = \begin{cases}
a &, \omega = \omega_0 \\
0 &, \omega \not= \omega_0
\end{cases}
\eeq
其形式更简单,但丢失了相位的信息。基本信号的更一般数学描述是复数形式:
\beq
s_\omega (t) = ae^{i (\omega t + \phi)}
\eeq
它包含了式$\cref{eq1}$描述的实况和虚部,具有更丰富的物理内含和更简洁的数学表达。
\par
一般周期性的信号(函数)都可以傅立叶级数展开:
\beq \label{eq4}
s(t) = \sum_m a_m e^{i(m \omega_0 t + \varphi)}
\eeq
因此,只要讲清楚一种基本形式的信号处理过程,就可推广到一般的信号处理过程。
\par
广义地说,不是待测信号本身的、对测量值的贡献都可视为噪声。它可分为来自外界的环境噪声,最典型的是市电噪声($50\mathrm{Hz}$);以及来自被测量对象本身的噪声,如热噪声。它们都混合待测信号的测量值中。
\par
从物理角度来看,任何待物理量都包含无规、随机的变化,这就是噪声。它是对有规律的信号的一种随机干扰,往往被扩展为与待测信号相冲突的无用信号。例如白噪声,是电子器件和电路中常见的一种噪声,电阻的热噪声,PN结的散弹噪声等都是白噪声,它在时域和频域都呈均匀分布。换句话说,对于一项实验,噪声可来自于实验对象本身,也可来自于测量系统, 包括传感器和测量仪器,更可能来自于实验系统以外的环境。
\par
对于有规律变化的环境噪声,可用\cref{eq4}式来表述:它的频率可能与待测信号的频率相同或相近,但更可能是其他频率。在时域可以看成频率不同、振幅随机的“信号”叠加(讲义~\cite{lec}中这里似乎有错误)
\beq
n(t) = \sum_\omega b(\omega) e^{i\omega t}
\eeq
其中振幅$b(\omega)$可能是常量(如市频干扰),亦可能是随机量$r(\omega)$;在频域中,随机噪声也可以理解为每一时刻的噪声电压频谱$n(\omega) $都不会是一样的\footnote{拟有误,随机噪声应该理解成一维随机高斯场。}

信噪比是科学与工程中所用的一种度量,用于比较所需信号的强度与背景噪声的强度,表 征噪声对信号的覆盖程度。
通过信号处理可提高信噪比,即提高了信号质量。为反映信号质量改善的程度,定义信噪改善比$SNIR$。
\beq
SNIR = \frac{SNR_0}{SNR_i}
\eeq
式中,$SNR_o$是经过处理(系统输出端)的信噪比,$SNR_i$是未经过处理(系统输入端)的信噪 比。$SNIR$越大,表明系统抑制噪声的能力越强。
\par
“微弱信号”不只意味着信号的绝对幅度很小,而是相对于噪声而言幅度很小,特别是那些被噪声淹没的信号,都称作微弱信号。只有在有效地抑制噪声的条件下放大微弱信号的幅度, 才能提取出真实的信号。因此,微弱信号检测所一门专门针对噪声的技术,其主要任务是提高信噪比。\par
提高信噪比的途径有两方面:一是通过物理屏蔽来抑制环境噪声,还可以通过不做实验来避免遭遇噪声,另一方面是通过信号处理滤除噪声。通常物理屏蔽的代价不少,关键是有另一类噪声在理论上不能通过物理屏蔽来抑制或隔断,它就是来自测量对象本身的噪声,尽管此类噪声在幅度上可能远低于环境噪声;因此,对于不能被屏蔽的噪声,信号处理就成为唯一的选择。\par
为了从噪声中提取出有用的信号,就需要研究噪声的来源与性质,分析噪声产生的原因和规律,以及噪声的传播途径,有针对性地采取有效措施抑制噪声,研究被测信号和噪声的统计特性及其差别,以寻找出从背景噪声中检测出有用信号的方法。
\subsubsection{滤波法}
\par
小信号是相对微弱信号信噪比更高的信号,可以采用{\bf 滤波}的方法进行信噪比的提升以实现精确测量。
\par
设输入信号为$s(t) + n(t)=a \cos(\omega_0 t+\varphi)+\sum_{-\infty}^{\infty}b(t,\omega)\cos(\omega t)$。带通滤波的作用是删去$(\omega_0 - \frac{\Delta \omega}{2} , \omega_0+ \frac{\Delta \omega}{2})$以外的信号,滤波后的总信号为
\beq
a \cos(\omega_0 t) +\sum_{\omega_0 -\Delta \omega/2}^{\omega_0 + \Delta\omega/2}b(t,\omega)\cos(\omega t)
\eeq
还可以使用低通滤波的方法,低通滤波的方法即设定一个截止频率,删去高频的部分,滤波之后信号为
\beq
A a\cos(\omega_0)t+ A\sum_0^{\omega_{\rm cut-off}}b(t,\omega)\cos(\omega t) = a\cos(\omega_0 t)+ n_l(t)
\eeq
其中$A$为放大倍数。可见,滤波后带通外的噪声仅在一个带通内表达(仅由式(D1-9)或(D1-
10)的右边第二项描述),从时域看噪声水平明显下降,信噪比得到明显改善。但带通内的噪声
不能被滤掉,且与信号一起同时被放大,我们称这部分噪声为同频噪声。
\par
实际中可以用示波器进行简单滤波,如\cref{filter}。
\cpicn{0.6}{filter}{$Q$值为 20 的滤波器滤波前后的信号:(a)、(b)分别为滤波前后的时域图;(c)、(d)分别为滤波前后的频域图。}
\subsubsection{零位法}
\par
另一种削弱噪声信号的方法为{\bf 零位法}(null method),零位法的本质是比较法测量。一般直接指示型测量仪表的测量方法是将被测信号放大到一定幅度,以驱动表头指针的偏转角度指示被测量的大小;或者经模数转换和数据处理后由液晶显示被测量的数值。而零位法是调整对比量的大小使其尽量接近被测量,由对比量指示被测量的大小,如\cref{d16}所示。图中的零位表指针只用来指示被测量和对比量的差异值,当零位表指示近似为零时,对比量的大小就表征了被测量的大小。对比量的调整可以手动实现,也可闭 环自动调整,如\cref{d16}中虚线所示。用这种方法测量的分辨率取决于对比量的调整和指示的分辨率。用机械测量类比,弹簧秤是一种直接指示仪表,而天平则是一种零位法仪表。
\par
由\cref{d16}可以看出,虽然被测量和对比量在传输过程中分别附加了干扰噪声$n_1(t)$和$n_2(t)$,但是在对比相减的过程中,$n_1(t)$和$n_2(t)$会在一定程度上相互抵消。两路信号传输过程越相似,$n_1(t)$和$n_2(t)$也会越近似,抵消作用越好。因此,与直接指示测量方法相比,零位法测量结果的信噪比更高,测量精度也更高。
\par
设被测信号
\beq
x(t) = s(t) + n_1 (t)
\eeq
对比量信号
\beq
x_r(t) = s_r(t) + n_2 (t)
\eeq
测量其差值得到
\beq
\Delta x (t) = s(t) - s_r(t) + n_1(t) - n_2(t)
\eeq
当调节对比量使得零位表计数为零时,$\Delta x = 0$,则$s(t) = n_1(t)-n_2(t)$,两组噪声相互抵消,则
\beq
s(t) \simeq s_r(t)
\eeq
可见,零位法更适合于抑制同源噪声(如环境噪声),但不适用于随机的热噪声。
\subsubsection{锁相放大法}
\par
锁相放大器(lock-in amplifier, LIA)也称锁定放大器,是通过特定的载波将湮没在噪声中的微弱信号提取出来的技术。锁相放大技术采用了频谱迁移和交流放大技术、用相敏检测器和低通滤波器来实现调制信号的解调。锁相放大器抑制噪声有三个基本出发点:
\begin{itemize}
\item 用调制器将直流或慢变信号的频谱\footnote{讲义此处有错字}迁移到调制频率$\omega_0$处,再进行放大,以避开传输过程所引入的以及放大器内部的$\frac{1}{f}$噪声的不利影响。
\item 利用相敏检测器实现被调制信号的解调过程,可以同时利用频率$\omega_0$和相角$\theta$进行检测,噪声与信号同頻又同相的概率很低。
\item 用低通滤波器来抑制宽带噪声。电路实现时,低通滤波器的频带可以做得很窄,而且 其频带宽度不受调制频率的影响。
\end{itemize}
锁相放大技术需要先对(慢变)信号频谱进行迁移,如\cref{d17},其调制过程是将低 频信号$V$乘以频率为$\omega_0$的正弦载波,从而将其频谱迁移到调制频率$\omega_0$的附近,之后进行选频放大,这样就不会把$\frac{1}{f}$噪声和低频漂移也放大了,如\cref{d17}(a)所 示,图中的虚线表示$\frac{1}{f}$噪声和白噪声的功率谱密度。经交流放大后,再用相敏检测器将其频谱迁移到直流($\omega = 0$)的两边,用窄带低通滤波器滤除噪声,在这里加一句废话应该看不见,就得到高信噪比的放大信号,如\cref{d17}(b)所示,图中用虚线表示低通滤波(LPF)的频率响应曲线。只要 LPF 的带宽足够窄,就能有效地改善信噪比。
\cpicn{0.6}{d17}{锁相放大器对信号频谱进行迁移的过程}
通过锁相放大技术的过程,相当于一个$Q$值极高的窄带滤波器,可以不受待测频率以外的噪声信号影响,从而使检测系统的性能大为改善。锁相放大器的等效噪声带宽可以达到$0.0001\mathrm{Hz}$,整体增益可以高达 1010 以上,即把$0.1\mathrm{nV}$的微弱信号可以放大到$1.0\mathrm{V}$以上,因而可以测量极其微小的信号。
\subsubsection{工作原理}
锁相放大器的基本结构如\cref{d19}所示的虚线框内,其中信号通道、参考通道为锁相放大器的输入通道,相敏检测器(PSD)和低通滤波器(LPF)等。对于三角函数信号,可直接从信号通道输入(虚线框内的)锁相放大器。对于非三角函数信号或慢信号(如直流信号),它在输入锁相放大器前,需要被与参考信号频率相同的信号所调制(如\cref{d19}虚线框外部分所示),参考信号既可以是外部输入,也可以是内部自带参考信号源提供。为不失一般性,以下数学推导以需要调制的非三角函数信号为出发点。
\cpicn{0.4}{d19}{锁相放大器工作原理(虚线框内)}
\par
设带噪声的待测信号
\beq
f(t)= s(t) + n(t)
\eeq
调制并放大$A_I$倍后的信号为
\beq
u(t) = A_I f(t) \sin \left(\omega_{0} t+\theta\right) = A_I s(t) \sin \left(\omega_{0} t+\theta\right) + A_I n(t) e\sin \left(\omega_{0} t+\theta\right)
\eeq
此时,信号与噪声都被同时放大了。
\par
锁相放大器采用参考信号$u_{r}(t)=\sin \left(\omega_{0} t\right)$解调,因调制信号是用参考信号通过某种物理机制触发或用参考信号本身,因此,参考信号与调制信号频率相同、相位差确定。
\bea
u_{p}(t)&=u_{r}(t) u_{\mathrm{s}}(t)=A_{I}[s(t)+n(t)] \sin \left(\omega_{0} t+\theta\right) \sin \left(\omega_{0} t\right) \\ &=\frac{1}{2} A_{I}[s(t)+n(t)]\left[\cos \theta-\cos \left(2 \omega_{0} t+\theta\right)\right]
\eea
低通滤波后留下了慢变的待测信号和同(低)频噪声:
\beq
u_{ox}(t)=\frac{1}{2} A_{I}\left[s(t)+n_{l}(t)\right] \cos \theta
\eeq
另一路相位相处$\frac{\pi}{2}$的信号经相同处理后得到
\beq
u_{o y}(t)=\frac{1}{2} A_{I}\left[s(t)+n_{l}(t)\right] \sin \theta
\eeq
于是被调制后信号(锁相放大器输入端信号)相对于解调信号的相位差
\beq
\theta=\arctan\frac{u_{o y}(t)}{u_{o x}(t)}
\eeq
输出信号振幅
\beq
u_{o}(t)=\sqrt{u_{\mathrm{ox}}^{2}(t)+u_{\mathrm{ov}}^{2}(t)}=\frac{1}{2} A_{I}\left[s(t)+n_{l}(t)\right]
\eeq
具有这种两互相垂直的信号输出、获得锁相放大器输入端完整信号能力的锁相放大器被称双相锁相放大器,本实验用的OE1022就是双相锁相放大器。信号的有效值
\beq
R=\sqrt{\frac{\sum_{0}^{N_{r}}\left\{\left[\left(s\left(t_{i}\right)+n\left(t_{i}\right)\right) \cos \left(\omega_{0} t\right)\right]^{2}+\left[(s(t)+n(t)) \sin \left(\omega_{0} t\right)\right]^{2}\right\}}{N_{T}}} \approx \frac{u_{o_{0} p p}(t)}{\sqrt{2} A_{t}}
\eeq
则锁相放大器输入信号的$x$、$y$分量分别为:
\bea \label{measure}
X&=R \cos \theta \approx \frac{u_{o x_{-} p p}(t)}{\sqrt{2} A_{I}} \\
Y&=R \sin \theta \approx \frac{u_{o y_{-p p}}(t)}{\sqrt{2} A_{I}}\\
R&=\sqrt{X^{2}+Y^{2}}
\eea
其中$R$、$X$、$Y$和$\theta$就是在锁相放大器屏幕上的显示值,是内部运算的结果;当噪声足够小、且$s(t)$为常量时,$R$为一常量。
\par
可见,锁相放大器用相敏检测器实现解调,并且实现既检幅又检相的功能。尽管锁相放大器已滤除了绝大部分噪声,但对同频噪声$n_l (t)$仍无能为力。并且由于放大后的信号仍包含噪声成分,等效于其相位差$\theta$也受随机噪声$n_l (t)$影响,从而存在相位噪声。在实际工作中,我们通常选择频谱中背景噪声最低的频率作为慢变信号的调制频率。\par
很多应用场合,锁相放大器的参考信号由外部提供(例如光学斩波器,信号发生器),此时叫做外部参考模式。锁相放大器的参考通道就会将外部输入的参考信号进行放大和整形,然后通过锁相环技术,在锁相放大器内部产生一个与外部输入参考信号频率相等、相位差锁定的的正弦信号。可见,锁相的含义是根据外部参考,利用锁相环产生一个频率相同、相位差锁定的参考信号。这个是锁相放大器能够进行后面相敏检测运算的基础。参考通道还有另一功能,即对参考输入进行移相处理,以使各种不同相移信号的检测结果达到最佳。
\par
如果采用内部参考模式,也就是锁相放大器自带的 SINE OUT 的正弦信号作为实 验的激励源,由于内部参考信号是锁相放大器自身产生的正弦波(与 SINE OUT 信号同频不同 相),因此频率及相位不需要锁相环获取,此时也就不用锁相;在这种情况下,由于频率及相 位是直接得到的,不需要放大和整形,测量的精度会更高。
\par

对于数字锁相放大器 Output 通道的输出信号,对应于原信号$R$或$X$、$Y$分量乘以其逻辑增益。电路实现时,LPF 的频带可以做得很窄,使锁相放大器达到较大的 SNIR。
\par
锁相放大器有几个重要参数:灵敏度,最小可测信号 MDS,时间常数,陡降,满刻度输出时的输入电平 FS,过载电平 OVL,输入总动态范围,输出动态范围,动态储备 DR,选频放大。锁相放大器的重要参数,影响着锁相放大器的工作。理解锁相放大器参数的含义及其(配置)对锁相放大器性能的影响,是正确使用锁相放大器的前提。
\begin{enumerate} 
\item 灵敏度从直观上来分析就是锁相放大器的当前显示量程,由信号通道的交流放大增益与相敏检波器解调增益决定。
\item 最小可测信号定义为输出能辨识的最小输入信号,可以理解为系统对小信号的分辨率。影响 MDS 的主
要因素有系统的内部噪声、温度漂移等,即结果会受到包括内部噪声、温度漂移的影响而产
生波动,MDS 定义为输出可以稳定在一定百分率波动下的最小输入。
\item 时间常数是低通滤波器$RC$电路的时间常数。可以简单地认为,时间常数越大,阶数越高,输出的带宽就越低,显示的测量幅度、相位等值就越稳定。然而,过大的时间常数会抹平输入信号(随时间)的变化,从而失去有用的信息。因此,在实际应用中,需要根据输入信号随时间变化的情况,协调时间常数与信噪比之间的平衡。
\item  FS 有时也称为满刻度灵敏度,它是用来表征锁相放大器测量灵敏 度的,拥有电压的量纲,与系统的总增益有关。
\item OVL 定义为锁相放大器任一级出现过载或临界过载时的输入电平。因为微弱信号检测通
常处理的是信噪比较低的输入,所以过载往往出现在噪声电压出现尖峰时。因此,可以将
OVL 理解为系统允许的最大输入噪声电压电平,即系统的最大噪声容限。应当指出,通常对
应不同的增益设置,各级放大情况将有所不同,其过载电平也会有所不同,故在指明 OVL
时,应当附带指出其 FS,这样才有意义。
\item  在给定 FS 的条件下(即给定的增益设置),锁相放大器的过载电平 OVL 与最小可测信
 号 MDS 比值的分贝数
 \beq
\text{输入总动态范围} = 20 \mathrm{lg\ } \frac{OVL}{MDS} \mathrm{dB}
\eeq
上文已经介绍,OVL 标称了锁相放大器的噪声容限,而 MDS 表明锁相放大器能够分辨的
最小信号。因而输入总动态范围可以理解为锁相放大器从噪声中提取有用信号的能力,即分
辨率越高,噪声容限越大,则输入总动态范围越大。
 \item 该参数定义为满刻度灵敏度 FS 与最小可测信号 MDS 比值的分贝数,即
 \beq
 \text{输出动态范围} = 20 \mathrm{lg\ } \frac{FS}{MDS} \mathrm{dB}
\eeq
 输出动态范围表示锁相放大器可以检测的有用输入信号的动态范围,即输入有效信号可以在该范围内波动而既不会导致锁相放大器不可分辨,也不会导致超过输出的最大范围。
 \item 动态储备 DR 定义为过载电平 OVL 与满刻度输出时的输入电平 FS 比值的分贝数
 \beq
 DR = 20 \mathrm{lg\ } \frac{OVL}{FS} \mathrm{dB}
\eeq
\par
锁相放大器输入信号在 PSD 处理之前需要交流放大,而在 PSD 处理之后是直流放大信
号即可。在总增益不变的情况下,如果调整交流增益增加,直流增益减小,则输入噪声经交流放大很容易使 PSD 过载,动态储备减小,同时输出的直流漂移减小。反之,如果增加直
流增益,降低交流增益,则动态储备提高,使锁相放大器具有良好的抗干扰能力,但以输出
稳定性为代价,降低了测量精度。\par
在确定的测量精度要求下,动态储备有最小值。精度要求越高,其最小值就越大。在模
拟锁相放大器中,低动态储备意味着更小的输出误差和漂移。在 OE1022 数字锁放中,高动
态储备不会增加输出误差和漂移,但是会增加输出噪声。然而,如果在 A/D 转换器前的模
拟放大器增益足够大,则其被放大的自身噪声比 A/D 转换器的噪声还大。这样,输出主要
受输入噪声影响。因此,增大模拟增益即减小动态储备并不能减小输出噪声。在分辨率要求极高的情况下,增益增大并不能提高信噪比,因此,这时可以降低增益从而提高动态储备。
\item 
 选频放大对某一段频率或单一频率的信号具有突出的放大作用,而对其他频率的信号具有较强抑制作用的放大。
 \end{enumerate}
\subsection{实验前思考题}
\subsubsection{实验一}
1. 请写出拍频信号的数学表达式,并分别用时域和频域图表述出来。

拍频信号的本质就是两个频率相近的信号的叠加,为了突出原理起见,我们假设两个信号振幅相同。设两个信号分别为
\bea
E_1 &=& E_0 \cos(\omega_1 t +\varphi_1) \\
E_2 &=& E_0 \cos(\omega_2 t +\varphi_2)
\eea
通过三角函数公式
\beq
\cos \alpha + \cos\beta = 2\cos\left(\frac{\alpha + \beta}{2}\right)\cos\left(\frac{\alpha - \beta}{2}\right)
\eeq
我们得到两列信号的叠加为
\beq\label{pai}
E = E_1 + E_2 = E_0 \cos(\omega t +\varphi_0 )\cos(\Delta \omega t/2+ \Delta \varphi/2)
\eeq
在上式中,我们定义了
\bea
\omega &=& \frac{\omega_1+\omega_2}{2}\\
\Delta \omega &=& \omega_1 - \omega_2 \\
\Delta \varphi &=& \varphi_2 - \varphi_1\\
\varphi_0 &=& \frac{\varphi_1 - \varphi_2}{2}
\eea
将上述信号做傅里叶变换即可得到频域图,利用公式
\beq
\int e^{i\omega t}e^{-i\omega^{\prime}t} dt = 2\pi \delta(\omega - \omega^{\prime})
\eeq
我们得到三角函数的傅里叶变换
\bea
c(\omega) &=& \int cos(\Omega t)e^{i\omega t} dt \\
&=& \int \frac{e^{i \Omega t}+e^{-i\Omega t}}{2} e^{i\omega t} \\
&=& 2\pi \frac{\delta(\Omega - \omega)+\delta(\omega+\Omega)}{2}
\eea
\cpicn{0.5}{pai1}{拍频时域图}
\cpicn{0.5}{pai2}{拍频频域图}
从式~\ref{pai}中我们可以看出,叠加后的信号一方面在做快速振动,快速震荡的角频率为$\omega$,另一方面快速振动的振幅又在做相对来说较慢的振动,这种现象成为拍。假设两个频率相近的信号,将他们相加后画出的图像如图~\cref{pai1}所示,将得到的图像做傅里叶变换,得到图~\cref{pai2}。如图~\cref{pai1},\cref{pai2}所示,我们确实可以看到叠加后的信号在做快速振动,将其做傅里叶变换到频域图中,发现两个信号的频率确实相近(相差远远小于他们的本身值。)

上面讨论仅仅是两列波实现的拍频现象,实际上,真正的信号由很多频率的三角函数叠加,如果这些信号的频率相近,也会得到上述拍频现象。

2. 噪声电压信号是个随机值,原则上,只要时间足够长,其平均值为零。如果待测信号是个恒定值,且环境噪声也是稳定的,则可以通过大量采样平均来获得待测信号值。
然而,实际测量中,信号不是常量,而是(相对)慢变量,而(环境)噪声也不稳定,
这样长时间大量采样平均的方法就不奏效,锁相放大器是个更好的解决方案。为更好
的使用锁相放大器,请回答以下问题:
1)如何用锁相放大器检测到待测的直流信号或慢变信号?(图 D1-9 中的 x(t)为直流
或慢变信号)
2)如用斩波器调制直流信号(如光强)
,被斩制后的信号(图 D1-9 中的$u_m(t)$ 信号)
仍然包含有直流分量(即平均值不为零),但该直流分量随交流信号输入锁相放
大器后不会被锁相放大器检测,请从数学推导上说明。
【注,一个理想的交流信号其平均值为零;一个平均值不为零的信号,其平均值可视为直
  流信号。】

1)检测到待测信号的手段有两个:
对于直流信号或慢变信号在输入到锁相放大器之前,待测信号会被与参考信号频率相同的信号调制,其频谱会迁移到调制频率附近,以避免放大器1/f噪声的影响,但待测信号仍然很弱,此时输入锁相放大器后会被放大器放大,以达到锁相放大器的最小可测信号的强度要求,接着输入相敏检测单元进行解调;解调信号由参考信号产生,与放大后的待测信号频率相同,相位差确定,对待测信号解调后再通过LPF,滤除宽带噪声,最后输出高信噪比的信号,也即检测到待测的直流信号或慢变信号。
2) 如果$u_m$仍然有直流分量$u_0$,设$u_m(t) = u_0+ x(t)\sin(\omega_0 t+ \theta)$,经过相敏检波器解调,假设解调信号为$A\sin(\omega_0 t)$,解调之后的信号为
\bea
u_p(t) &=& Ax(t) \sin(\omega_0 t +\theta)\sin(\omega_0 t)+A\sin(\omega_0 t)\\
&=& \frac{1}{2}(\cos \theta  - \cos(2\omega_0 t+\theta)) +Au_0 \sin(\omega_0 t)
\eea
解调后的信号会经过低通滤波器,大于频率阈值$\omega_{\rm cut-off}$的噪声会被除去,因此$u_p(t)$中频率为$2\omega_0,\omega_0$的噪声会被除去,不会对最终测得的相位差以及幅度等参量产生影响。



3. 相位角以及相位差的含义是什么?锁相放大器输出的是待测信号的相位角还是待测
信号与参考信号之间的相位差?

如果某一物理量随时间(空间)做正弦变化或余弦变化,决定该物理量在某一时刻相位角是在某一特定时刻(位置)决定信号状态的一个数值,即为相位角;以正弦波为例(实际上,任何一个信号都可以分解成很多三角函数的叠加)
\beq
x(t) = x_0 \cos(\omega t+\varphi)
\eeq
这里$\omega t +\varphi$就被称作相位
在$t=0$ 时的数值也被称为初相位角,这里的初始相位角是$\varphi$
考察频率相同的两个物理量,则二者之间相位差是二者的相位角的差。以时域
中的物理量为例,设二者角频率为$\omega$,
\bea
u_1 &=& A_1 \sin(\omega t +\varphi_1)\\
u_2 &=& A_2 \sin(\omega t +\varphi_2)
\eea
则两个信号的相位差为
\beq
\Delta \varphi = \varphi_2 - \varphi_1 +\omega \Delta t
\eeq
如果两个物理量所处的时刻相同,相位差就等于初始相位角的差,相位角收到时间、角频率、初始相位差的影响。

实际上,在测量中,只能测量到实际待测信号与参考信号的相位差,绝对的相位没有意义(因为相位的零点可以任意选取)。在本实验中,锁相放大器输出的是待测信号与参考信号的相位差,它收到调制信号与解调信号相位差的影响。


4. 请预测一根长 1.0 米的同轴电缆会给相位差的测量带来多大的影响?实验能测出来吗?多高的频率才能测出来?

根据介质中的麦克斯韦方程组(例如,参见~\cite{electromagnetic})
\bea
\nabla \cdot \vec{E} &=& \rho/\epsilon \\
\nabla \times \vec{E} &=& -\frac{\partial \vec{B}}{\partial t} \\
\nabla \cdot \vec{B} &=& 0\\
\nabla \times \vec{B} &=& \mu\vec{j} + \mu\epsilon \frac{\partial E}{\partial t}
\eea
利用公式
\beq
\nabla\times( \nabla\times \vec{E} ) =  \nabla(\nabla \cdot \vec{E}) - \nabla^2 \vec{E}
\eeq
将麦克斯韦方程组带入上式两边,利用欧姆定律,并取$\rho = 0 $,得到
\beq
\frac{\partial^2 \vec{E} }{\partial t^2}+\mu \sigma \partial_t \vec{E} - \frac{1}{\mu\epsilon} \nabla^2 \vec{E} =0
\eeq
上式实际上是一个波动方程,$\mu \sigma \partial_t \vec{E}$可以看做阻尼项,如果$\sigma$很小,则阻尼很小,实际上阻尼意味着一个弛豫时间,如果知道输入信号的强度和锁相放大器接受的信号强度,可以测量这个弛豫时间,进而测量金属电导率。所以我们得到,电磁场在电缆中传播的速度为
\beq
v = \frac{1}{\mu \epsilon} \simeq 10^8 \,\mathrm{m/s}
\eeq
因为电缆有$1$m长,所以在电缆的传播过程导致了时间差
\beq
\Delta t = \frac{l}{v}
\eeq
进而造成了相位改变
\beq
\omega \Delta t = 2\pi \nu \Delta t
\eeq
所以当相位改变大于相位分辨率时,就能测出由上述效应导致的相位差变化。实验中用到的锁相放大器分辨率约为$0.01^{\circ}$,由上面的条件计算得到,信号的频率要大于$3\mathrm{kHz}$才能测出相位差。

6.(选)用复数取代三角函数,将锁相放大器的工作过程描述一遍;

设带噪声的待测信号
\beq
f(t)= s(t) + n(t)
\eeq
调制并放大$A_I$倍后的信号为
\beq
u(t) = A_I f(t) \sinomega = A_I s(t)\sinomega + A_I n(t) \sinomega
\eeq
此时,信号与噪声都被同时放大了。
\par
锁相放大器采用参考信号$u_{r}(t)=e^{i \left(\omega_{0} t\right)}$解调,因调制信号是用参考信号通过某种物理机制触发或用参考信号本身,因此,参考信号与调制信号频率相同、相位差确定。
\bea
u_{p}(t)&=&u_{r}(t) u_{\mathrm{s}}(t)=A_{I}[s(t)+n(t)] \sinomega \frac{e^{i\omega_0 t}-e^{-i\omega_0 t}}{2} \\ &=&\frac{1}{2} A_{I}[s(t)+n(t)]\left[\frac{e^{i \theta}+e^{-i\theta}}{2}- \frac{e^{i(2\omega_0 t+\theta)}+e^{-i(2\omega_0 t+\theta)}}{2}\right]
\eea
低通滤波后留下了慢变的待测信号和同(低)频噪声:
\beq
u_{ox}(t)=\frac{1}{2} A_{I}\left[s(t)+n_{l}(t)\right] \frac{e^{i \theta}-e^{-i\theta}}{2}
\eeq
另一路相位相处$\frac{\pi}{2}$的信号经相同处理后得到
\beq
u_{o y}(t)=\frac{1}{2} A_{I}\left[s(t)+n_{l}(t)\right] \frac{e^{i \theta}-e^{-i\theta}}{2}
\eeq
于是被调制后信号(锁相放大器输入端信号)相对于解调信号的相位差
\beq
\theta=\arctan\frac{u_{o y}(t)}{u_{o x}(t)}
\eeq
输出信号振幅
%\beq
%u_{o}(t)=\sqrt{u_{\mathrm{ox}}^{2}(t)+u_{\mathrm{ov}}^{2}(t)}=\frac{1}{2} A_{I}\left[s(t)+n_{l}(t)\right]
%\eeq
具有这种两互相垂直的信号输出、获得锁相放大器输入端完整信号能力的锁相放大器被称双相锁相放大器,本实验用的OE1022就是双相锁相放大器。信号的有效值
\beq
R=\sqrt{\frac{\sum_{0}^{N_{r}}\left\{\left[\left(s\left(t_{i}\right)+n\left(t_{i}\right)\right) \cos \left(\omega_{0} t\right)\right]^{2}+\left[(s(t)+n(t)) \sin \left(\omega_{0} t\right)\right]^{2}\right\}}{N_{T}}} \approx \frac{u_{o_{0} p p}(t)}{\sqrt{2} A_{t}}
\eeq
则锁相放大器输入信号的$x$、$y$分量分别为:
\bea 
X&=&R \cos \theta \approx \frac{u_{o x_{-} p p}(t)}{\sqrt{2} A_{I}} \\
Y&=&R \sin \theta \approx \frac{u_{o y_{-p p}}(t)}{\sqrt{2} A_{I}}\\
R&=&\sqrt{X^{2}+Y^{2}}
\eea
其中$R$、$X$、$Y$和$\theta$就是在锁相放大器屏幕上的显示值,是内部运算的结果;当噪声足够小、且$s(t)$为常量时,$R$为一常量。
\par
可见,锁相放大器用相敏检测器实现解调,并且实现既检幅又检相的功能。尽管锁相放大器已滤除了绝大部分噪声,但对同频噪声$n_l (t)$仍无能为力。并且由于放大后的信号仍包含噪声成分,等效于其相位差$\theta$也受随机噪声$n_l (t)$影响,从而存在相位噪声。在实际工作中,我们通常选择频谱中背景噪声最低的频率作为慢变信号的调制频率。

7. (选)用付立叶变换描述滤波过程;帮助理解图 D1-5(时域与频域)背后的数学。
【示波带快速付立叶变换(FFT)功能:MATH→操作(选 FFT)】
\subsubsection{实验二}

\newpage
\section{实验步骤与记录}
\begin{tabular}{|p{8em}|p{8em}|p{8em}|p{8em}|}
	\hline 
	专业:     &Physics       &年级:      & 17     \\
	\hline
	姓名:& 徐昊霆 &学号:&17353071  \\
	\hline
	室温:&                    &实验地点 & 教学楼 \\
	\hline	
	学生签名: & & 评分: & \\
	\hline
	日期: & 2019.9.9 & 教师签名:&  \\
	\hline
\end{tabular}
\subsection{实验内容、步骤、结果}
\subsubsection{锁相放大器的基本操作和基本参数}
	\subsection{锁相放大器的基本操作和基本参数——工作原理的体验}
	\subsubsection{测量信号R、$\theta$、X 以及 Y 值,并验证它们之间的关系}
	OE1022的内部振荡器SINE OUT相当于函数发生器,输出正弦信号。现在用它产生一个幅值为 80mVrms、频率约为 1kHz 的正弦波,并用 OE1022 进行测量。
\begin{enumerate}[(1)]
\item  断开所有与机箱连接的信号线,接入电源,打开电源开关,此时系统处于默认设置状态。若如果系统不是默认设置状态,可以在前面板上选择 SAVE RECALL 菜单,Save\&Recall 设置为 recall,此时 Channel 会变成 Default,按下软键 3(见图 \ref{D1-14}),将Execute 设置为 YES,即可恢复默认设置状态。
		
若已经恢复默认设置状态,在前面板上选择 REF/PHASE 子菜单,"Ref.source" 设置为“Internal”;Ref.frequency 为默认值“1.000kHz”;选择进入“Sine Output”下级菜单,扫描类型 "Sweep Type" 设置为“Fixed”,通过数字键盘在 "Voltage" 中输入“0.08”,得到幅值为 80mVrms、频率为 1.000kHz 的正弦信号.
		
\item 用一条 BNC-BNC 信号线连接 OE1022 前面板 SINE OUT 输出接口和 SIGNAL IN 的 A/I接口,如图\ref{D1-12}所示:
		\begin{figure}[H]
			\centering
			\includegraphics[scale=0.5]{D1-12}
			\caption{测量信号线连接图}\label{D1-12}
		\end{figure}
		
		\item 观察主界面中监测栏的 Overload 是否提示溢出:
		\begin{figure}[H]
			\centering
			\includegraphics[scale=0.5]{D1-13}
			\caption{锁相放大器显示屏主界面监测栏}\label{D1-13}
		\end{figure}
	
		\item 若前级输入溢出,则显示 Overload: INPUT NONE;若放大溢出,则显示 Overload: NONE GAIN;若同时溢出,则显示 Overload: INPUT GAIN。
		
		\item 前级溢出时应立即减小数字信号发生器输出幅值,放大溢出应立即调节灵敏度(sensitivity)8值(OE1022输入端峰值高于1.7V或谷值低于-1.7V时发生前级溢出,且默认灵敏度值为100mV,因此本例中数字信号发生器输出幅值为 80mVrms 的正弦波时不会发生溢出,但是测量其他信号时要注意溢出情况)。调节灵敏度值的方法见下:
		
		\item 调节灵敏度值。按下前面板 GAIN/TC 按键进入子菜单。
		\begin{figure}[H]
			\centering
			\includegraphics[scale=0.5]{D1-14}
			\caption{GAIN/TC 菜单位置}\label{D1-14}
		\end{figure}
	
		\item GAIN/TC 子菜单界面如图\ref{D1-15}。
		\begin{figure}[H]
			\centering
			\includegraphics[scale=0.5]{D1-15}
			\caption{GAIN/TC 菜单参数显示}\label{D1-15}
		\end{figure}
		
		\item 按下软键 1 以选中 Sensitivity 功能,选中区域会有高亮显示,通过旋转旋钮调节 Sensitivity值,使测量信号值尽量满偏但不超量程。此处我们调节为 100mV 即可。至此,我们即简单测出了从函数信号发生器输出的正弦波幅值大小以及相位(测量结果参考值:R=80.08mV,$\theta=-72.78^\circ$,如果相位差显示其他值,可以点击ref 设signal 按软键1进行调节ref phase),比较示波器的读数和锁相放大器的 R 值,以理解 R 值的确切含义
		
		\item 主界面数据栏显示 R、$\theta$、X 及 Y 值。按下前面板 DISPLAY 按键进入子菜单
		\item DISPLAY 子菜单界面如下
		\begin{figure}[H]
			\centering
			\includegraphics[scale=0.5]{D1-16}
			\caption{DISPLAY 菜单界面}\label{D1-16}
		\end{figure}
	
		\item 系统默认设置中,数据栏上方显示 R,下方显示 θ 值,通过以下介绍的方法可更改显示的数值:例如将上方显示的 R 值更改为采用 XY 坐标来显示 θ 值的方法:首先按软键 2,使其选中Top;再按软键 3 选中 Type,Type 区域此时高亮显示,通过调节旋钮可选择 Chart(XY 坐 标)或 Bar(数字百分比),我们选择 Chart;再按软键 3 选 Trace,Trace 区域此时高亮显示,通过调节旋钮可选择显示 R、$\theta$、X、Y,我们选择 $\theta$。然后按下前面板右下角处的 START CONT 按键,此时图像开始生成
		
		\item 通过改变$\theta$值,观察 X、Y 是如何相应地改变
		\begin{table}[H]
			\centering
			\caption{$\theta$与X,Y关系}
			\begin{tabular}{| c | p{2cm} | p{2cm} | p{2cm} |}
				\hline
				次数 & $\theta$ & X(mV) & Y(mV) \\
				\hline
				1 &  &  &  \\
				\hline
				2 &  &  &  \\
				\hline
				3 &  &  &  \\
				\hline
				4 &  &  &  \\
				\hline
				5 &  &  &  \\
				\hline
				6 &  &  &  \\
				\hline
			\end{tabular}
		\end{table}
		
		\item 可以更改检测栏的内容使其实时显示 R、$\theta$、X、Y 值。方法:按下前面板 DISPLAY 按键进入子菜单,再按软键 1 将 Monitor 设置中从 Setting 切换为 Output,此时监测栏显示 R、$\theta$、 X 和 Y 的值
		
		\item 影响相位差测量的因素
		改变不同的频率值,看看 R、X、Y、和$\theta$分别有什么变化?为什么?[提示,可用 sweep 设置参考信号。
		\begin{table}[H]
			\centering
			\caption{$\theta$与X,Y关系}
			\begin{tabular}{|  p{2cm} | p{2cm} | p{2cm} | p{2cm} | p{2cm} |}
				\hline
				频率 & $\theta$ & X(mV) & Y(mV) & R \\
				\hline
				 &  &  &  & \\
				\hline
				 &  &  &  & \\
				\hline
				 &  &  &  & \\
				\hline
				 &  &  &  & \\
				\hline

			\end{tabular}
		\end{table}
		
		
		\item ) 在 1 kHz 至 100kHz 范围内观察一米电缆对相位差的影响。
		设置某一参考信号频率,记录其相位差$\theta_0(f)$值,然后延长
		sine out 到 signal in 的电缆,记录延长电缆后的相位差$\theta_e(f)$)值,并计算$\delta\theta(f)=\theta_e(f)-\theta_0(f)$
		\begin{table}[H]
			\centering
			\caption{$\theta$与X,Y关系}
			\begin{tabular}{|  p{2cm} | p{2cm} | p{2cm} | p{2cm} |}
				\hline
				频率 & $\theta_0(f)$ & $\theta_e(f)$ & $\delta\theta(f))$ \\
				\hline
				&  &  &   \\
				\hline
				&  &  &   \\
				\hline
				&  &  &   \\
				\hline
				&  &  &   \\
				\hline
				
			\end{tabular}
		\end{table}
		操作验证:是否可以先测量不同频率下的一组$\theta_0(f)$值,然后再延长电缆,再测量对应频率下的一组$\theta_e(f)$)值?能定性解释$\theta_0(f)$吗?
\end{enumerate}
		\subsubsection{不同时间常数、陡降、动态储备下观测滤波器效果}
		\begin{enumerate}[(1)]
		\item 断开所有与机箱连接的信号线,接入电源,打开电源开关,此时系统处于默认设置状态。
		在前面板上选择 REF/PHASE 子菜单,Ref.source 设置为“Internal”;Ref.frequency 为默认
		值“1.000kHz”;选择进入“Sine Output”下级菜单,扫描类型 Sweep Type 设置为“Fixed”,通
		过数字键盘在 Voltage 中输入“0.05”,得到幅值为 50mVrms、频率为 1.000kHz 的正弦信号。
		\item 用 BNC-BNC 信号线连接 OE1022 前面板 SINE OUT 输出接口和 SIGNAL IN 的 A/I 接口,
		OE1022 前面板 CH1 接口和示波器输入接口。
		\item 调节灵敏度值,本例中数字信号发生器输出有效值为 50mVrms 的正弦波,相位(phase)
		设置为 60deg,灵敏度设置为 100mV 即可。
		\item 按下 OE1022 前面板的 GAIN/TC 按键,调节时间常数和陡降,设置时间常数为 30μs,陡
		降为 6dB/oct。
		\item 按下 OE1022 前面板 CHANNEL OUTPUT 按键,进入子菜单,在 Output 选项中选择 CH1, 在 Speed 选项中选择 Fast,Source 选项默认为 R(Fast 模式下输出源可选择 R、X、Y,
		Slow 模式下输出源可选择 R、X、Y、Rh1、Xh1、Yh1、Rh2、Xh2、Yh2、Noise、$\theta$、$\theta$h1、 $\theta$h2),Offset\&Expand 中默认 Offset 为 0,Expand 为 1。
		\item 在示波器上调节时基、输入通道设置等,可观测到在 30μs 时间常数和 6dB/oct 陡降下channel out 输出 R 值波形.(示波器所测量的 CH1 输出信号频率与设定信号频率是什么关系?时间常数大小对此
		有何影响?为什么?)
		
		\item 在 OE1022 的 Channel out 菜单中设置 Source 为 X、Y,在示波器上观测
		
		\item 在 OE1022 的 Gain/TC 菜单中,对时间常数进行改变,时间常数可选择 10$\mu$s、30$\mu$s、100$\mu$s、
		300$\mu$s……10s、30s,通过增大时间常数的值能够使系统的输出端更稳定,也能减轻输入端
		噪声对输出端的影响。时间常数除了对系统的稳定性和精度有影响外,还会影响系统的响
		应时间,时间常数还决定噪声测量时的等效噪声带宽(ENBW)。在此特别说明一下,等效
		噪声带宽指的并不是滤波器的-3 dB 带宽,它指的是对高斯噪声10的有效带宽。观察示波
		器波形,并给出:时间常数为多少个信号周期时,才能观察到稳定的直流输出?
		
		\item 在 OE1022 的 Gain/TC 菜单中,对滤波器陡降进行改变。在同样的测量准确度下,请分别
		选择 6dB/oct、12dB/oct、18dB/oct、24dB/oct 四档,在 TC 不大于 30$\mu$s 的情况下,观察示
		波器波形的变化。
\end{enumerate}
		\subsubsection{制定方案,观测不同时间常数下,输出信号对输入信号变化的响应。}
		\begin{enumerate}
			\item 分别设置时间常数:30$\mu$s、100$\mu$s、
			300$\mu$s、10ms、100ms、300ms
			\item 改变 SINE OUT 幅值设定,每个时间常数改变幅值为$V_rms$从 $50mV_rms$和$70mV_rms$
			\item 用“START CONT”激活 chart 图以观察改变前后的曲线来确定其输出响应,并记录图形
		
		\item 在 OE1022 的 Gain/TC 菜单中,对滤波器陡降进行改变。在同样的测量准确度下,分别选择 6dB/oct、12dB/oct、18dB/oct、24dB/oct 四档,观察在输入信号改变时,输出信号对输入信号变化的响应;并理解滤波器陡降的作用
	\end{enumerate}
	\subsection{强噪声背景下的弱信号检测——应用体验}
	\subsubsection{用示波器观察教学实验箱噪声发生器输出的噪声是否白噪声?记录时域和频域的图谱来说明。}
	\subsubsection{不同的噪声水平下,考察 OE1022 如何选用合适的动态储备测量信号?}
	在 OE1022 的 Gain/TC 菜单中,对动态储备进行设置,可设置为 Low、High、Normal
	三档之一。动态储备表示锁相放大器对噪声容忍程度的大小,OE1022 动态储备可大于
	100 dB,高的动态储备会产生输出噪声和漂移,当动态储备较高时,由于模数转换器的噪声存在导致输出误差增加。由于所有的信号源都存在本底噪声,固在 PSD 提取信号过程
	中就会掺杂着噪声,如果噪声很大,在高动态储备测量中就会产生较大的输出误差。如果
	外部噪声较小,则其输出主要是受 OE1022 自身噪声影响。这时可以通过降低动态储备和
	直流增益可以减小输出误差。因此,在实际应用中应尽量使用低动态储备。默认配置下系
	统会根据输入信号的 R 值自动调整所需要的最小动态范围。观察示波器波形变化,给出结
	论。
	\subsubsection{比较 OE1022 与示波器对不同信号强度和信噪比的测量结果}
	\begin{enumerate}
		\item 使用 OE1022 产生频率 1kHz,幅值为 100mVrms(0.282Vpp)的正弦波信号;
		\item 对锁相放大器 OE1022 进行以下设置:
		
		进入 INPUT FLTERS 菜单,设置 Source 为 A;
		
		进入 GAIN TC 菜单,设置 Sensitivity 为 500mV,Reserve 为 Normal,Time Constant
		为 1s,Filter dB/oct 为 24dB;
		
		进入 REF PHASE 菜单,设置 Ref. source 为 Internal;
		\item 使用 BNC-BNC 信号线连接 OE1022 的“SINE OUT”接口与实验箱本实验框图中的
		“VIN”接口;
		\item 通过分线器,使用 BNC-BNC 信号线将实验箱本实验框图中的“VOUT”接口分别连接
		到 OE1022 的“A/I”接口、和示波器输入端;
		\item 读取 OE1022 测到的 R 值,即为被噪声信号淹没的正弦信号有效值;
		\item 改变 OE1022 产生正弦波有效值,在不同信噪比下重复上述测量;
		
		\begin{table}[H]
			\centering
			\begin{tabular}{|  p{2cm} | p{2cm} | p{2cm} | p{2cm} | p{2cm} | p{2cm} |}
				\hline
				频率 & 正弦波$V_in$幅值(m$V_rms$) & 噪声信号大小(m$V_rms$) & 信噪比(dB)&锁相放大器测量R(m$V_rms$) & 示波器测量值(m$V_rms$)\\
				\hline
				&  &  & & &  \\
				\hline
				&  &  &  & & \\
				\hline
				&  &  &  & & \\
				\hline
				&  &  &  & & \\
				\hline
				&  &  &  & & \\
				\hline
				&  &  &  & & \\
				\hline
				
			\end{tabular}
		\end{table}
	\end{enumerate}
\subsection{实验过程中遇到的问题记录}
\newpage
\section{分析与讨论}
\begin{tabular}{|p{7em}|p{7em}|p{7em}|p{7em}|}
	\hline 
	专业:     &Physics       &年级:      & 17     \\
	\hline
	姓名:& 徐昊霆 &学号:&17353071  \\
	\hline
	日期&  \today              & &  \\
	\hline	
	评分 & & 教师签名 & \\
	\hline
\end{tabular}


\subsection{实验后思考题}

\bibliographystyle{siam}
\bibliography{cites}
\end{document}
