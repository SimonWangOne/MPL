\documentclass[utf8]{article}
\input{macros.tex}

%\cpic{<尺寸>}{<文件名>}}用于生成居中的图片。
\newcommand{\cpic}[2]{
\begin{center}
\includegraphics[scale=#1]{#2}
\end{center}
}

%\cpicn{<尺寸>}{<文件名>}{<注释>}用于生成居中且带有注释的图片,其label为图片名。
\newcommand{\cpicn}[3]
{
\begin{figure}[h!]
\cpic{#1}{#2}
\caption{#3\label{#2}}
\end{figure}
}
\title{实验CC1+A 热辐射设计性实验}
\begin{document}
\maketitle  
	\begin{tabular}{|p{8em}|p{8em}|p{8em}|p{10em}|}
		\hline
		\large{实验方案} &\large{实验记录}  &\large{分析讨论} &\large{总成绩}\\
		\hline
		        &          &          &  \\
	    \hline
	\hline 
	年级、专业: &17级物理学 &组号:& 7+A \\
	\hline
	姓名:& 徐昊霆、谢梓冰 &学号:&17353071、17353070  \\
	\hline
	日期:& 2019.6.2 &教师签名: &  \\
    \hline	
        \end{tabular}

\newpage
\tableofcontents
\section{实验原理与方案}

\subsection{实验目的}
\par 
1.传感器辐射强度(辐射通量、或能流密度)校正,\textbf{获得热辐射传感器输出信号与当地热辐射强度的定量关系式。}
\par
2.比较带透镜的热辐射传感器(SMTIR9902sil)与不带透镜的传感器(SMTIR9902)之间的结果,\textbf{获得带透镜的热辐射传感器输出信号与辐射表面温度的定量关系},分析原因并给出其适用范围和使用规范。
\par 
3.校正辐射体表面温度(具体校正什么下面会说明)。
\par
4.选做:测量不同物体的防辐射能力。
\subsection{仪器用具}
		\begin{tabular}{c|c|c|c}
			\hline
			编号 & 仪器名称 &数量& 主要参数(型号,测量范围,精度) \\
			\hline 
			1 &  黑体辐射与红外测量装置 &1 &DHRH-B;含标尺(60cm),位移导轨\\
			& & & 辐射器(三种辐射面)、热辐射传感器(SMTIR9902)\\
			2 & 数字多用表&2 & RIGOL DM3058E \\
			3 & 程序控制电源 & 1 &RIGOL DP832 \\
			4 & 计算机& 1 & 以安装LabView和控温软件,电脑进入win7-64位系统 \\
			5&透明塑料包装膜 &1& 带透镜传感器的外包装 \\
			6&螺旋测微仪&1& 精度为0.001\\
			7&游标卡尺&1& 50分度,内部含有润滑油充分润滑\\
			8&米黄色纸张&1& B5,购买于淘宝\\
			9&剪刀&1& 国产,放置于讲台的笔筒中\\
			\hline

\end{tabular}


                
\newpage
\section{实验步骤与记录}
\begin{tabular}{|p{9em}|p{9em}|p{9em}|p{9em}|}
	\hline 
	专业:     &Physics       &年级:      & 17     \\
	\hline
	姓名:& 徐昊霆 &学号:&17353071  \\
	\hline
	室温:&                    &实验地点 & 教学楼A512 \\
	\hline	
	学生签名: & & 评分: & \\
	\hline
	日期: & 2019.5 & 教师签名:&  \\
	\hline
\end{tabular}

\newpage
\section{分析与讨论}
\begin{tabular}{|p{9em}|p{9em}|p{9em}|p{9em}|}
	\hline 
	专业:     &Physics       &年级:      & 17     \\
	\hline
	姓名:& 徐昊霆、谢梓冰 &学号:&17353071、17353070  \\
	\hline
	日期&     2019.5.31               & &  \\
	\hline	
	评分 & & 教师签名 & \\
	\hline
\end{tabular}

\end{document}
