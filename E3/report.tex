\documentclass{article}
\usepackage{xeCJK}
%\setCJKmainfont{AR PL UKai CN}
%\setCJKmainfont{AR PL UKai}
\usepackage{geometry}
\usepackage{caption}
\usepackage{graphicx, subfig}
\geometry{a4paper,left=4cm,right=4cm}
\usepackage{appendix}
\usepackage{amsmath}
\usepackage{amssymb,color}
\usepackage[colorlinks,linkcolor=blue,anchorcolor=blue,citecolor=red]{hyperref}
\usepackage{slashed}
\usepackage{simplewick}
\usepackage{tikz}
\usepackage{tcolorbox}
%colors
\def\blacktext#1{{\color{black}#1}}
\def\bluetext#1{{\color{blue}#1}}
\def\redtext#1{{\color{red}#1}}
\def\darkbluetext#1{{\color[rgb]{0,0.2,0.6}#1}}
\def\skybluetext#1{{\color[rgb]{0.2,0.7,1.}#1}}
\def\cyantext#1{{\color[rgb]{0.,0.5,0.5}#1}}
\def\greentext#1{{\color[rgb]{0,0.7,0.1}#1}}
\def\darkgray{\color[rgb]{0.2,0.2,0.2}}
\def\lightgray{\color[rgb]{0.6,0.6,0.6}}
\def\gray{\color[rgb]{0.4,0.4,0.4}}
\def\blue{\color{blue}}
\def\red{\color{red}}
\def\green{\color{green}}
\def\darkgreen{\color[rgb]{0,0.4,0.1}}
\def\darkblue{\color[rgb]{0,0.2,0.6}}
\def\skyblue{\color[rgb]{0.2,0.7,1.}}
%%control
\def\be{\begin{equation}}
\def\ee{\nonumber\end{equation}}
\def\beq{\begin{equation}}
\def\eeq{\end{equation}}
\def\bea{\begin{eqnarray}}
\def\eea{\end{eqnarray}}
\def\bmat#1{\left(\begin{array}{#1}}
\def\emat{\end{array}\right)}
\def\bcase#1{\left\{\begin{array}{#1}}
\def\ecase{\end{array}\right.}
\def\bmini#1{\begin{minipage}{#1\textwidth}}
\def\emini{\end{minipage}}
\def\tbox#1{\begin{tcolorbox}#1\end{tcolorbox}}
\def\pfrac#1#2#3{\left(\frac{\partial #1}{\partial #2}\right)_{#3}}
%%symbols
\def\bropt{\,(\ \ \ )}
\def\sone{$\star$}
\def\stwo{$\star\star$}
\def\sthree{$\star\star\star$}
\def\sfour{$\star\star\star\star$}
\def\sfive{$\star\star\star\star\star$}
\def\rint{{\int_\leftrightarrow}}
\def\roint{{\oint_\leftrightarrow}}
\def\stdHf{{\textit{\r H}_f}}
\def\deltaH{{\Delta \textit{\r H}}}
\def\ii{{\dot{\imath}}}
\def\skipline{{\vskip0.1in}}
\def\skiplines{{\vskip0.2in}}
\def\lagr{{\mathcal{L}}}
\def\hamil{{\mathcal{H}}}
\def\vecv{{\mathbf{v}}}
\def\vecx{{\mathbf{x}}}
\def\vecy{{\mathbf{y}}}
\def\veck{{\mathbf{k}}}
\def\vecp{{\mathbf{p}}}
\def\vecn{{\mathbf{n}}}
\def\vecA{{\mathbf{A}}}
\def\vecP{{\mathbf{P}}}
\def\vecsigma{{\mathbf{\sigma}}}
\def\hatJn{{\hat{J_\vecn}}}
\def\hatJx{{\hat{J_x}}}
\def\hatJy{{\hat{J_y}}}
\def\hatJz{{\hat{J_z}}}
\def\hatj#1{\hat{J_{#1}}}
\def\hatphi{{\hat{\phi}}}
\def\hatq{{\hat{q}}}
\def\hatpi{{\hat{\pi}}}
\def\vel{\upsilon}
\def\Dint{{\mathcal{D}}}
\def\adag{{\hat{a}^\dagger}}
\def\bdag{{\hat{b}^\dagger}}
\def\cdag{{\hat{c}^\dagger}}
\def\ddag{{\hat{d}^\dagger}}
\def\hata{{\hat{a}}}
\def\hatb{{\hat{b}}}
\def\hatc{{\hat{c}}}
\def\hatd{{\hat{d}}}
\def\hatN{{\hat{N}}}
\def\hatH{{\hat{H}}}
\def\hatp{{\hat{p}}}
\def\Fup{{F^{\mu\nu}}}
\def\Fdown{{F_{\mu\nu}}}
\def\newl{\nonumber \\}
\def\vece{\mathrm{e}}
\def\calM{{\mathcal{M}}}
\def\calT{{\mathcal{T}}}
\def\calR{{\mathcal{R}}}
\def\barpsi{\bar{\psi}}
\def\baru{\bar{u}}
\def\barv{\bar{\upsilon}}
\def\qeq{\stackrel{?}{=}}
\def\torder#1{\mathcal{T}\left(#1\right)}
\def\rorder#1{\mathcal{R}\left(#1\right)}
\def\contr#1#2{\contraction{}{#1}{}{#2}#1#2}
\def\trof#1{\mathrm{Tr}\left(#1\right)}
\def\trace{\mathrm{Tr}}
\def\comm#1{\ \ \ \left(\mathrm{used}\ #1\right)}
\def\tcomm#1{\ \ \ (\text{#1})}
\def\slp{\slashed{p}}
\def\slk{\slashed{k}}
\def\calp{{\mathfrak{p}}}
\def\veccalp{\mathbf{\mathfrak{p}}}
\def\Tthree{T_{\tiny \textcircled{3}}}
\def\pthree{p_{\tiny \textcircled{3}}}
\def\dbar{{\,\mathchar'26\mkern-12mu d}}
\def\erf{\mathrm{erf}}
\def\const{\mathrm{constant}}
\def\pheat{\pfrac p{\ln T}V}
\def\vheat{\pfrac V{\ln T}p}
%%units
\def\fdeg{{^\circ \mathrm{F}}}
\def\cdeg{^\circ \mathrm{C}}
\def\atm{\,\mathrm{atm}}
\def\angstrom{\,\text{\AA}}
\def\SIL{\,\mathrm{L}}
\def\SIkm{\,\mathrm{km}}
\def\SIyr{\,\mathrm{yr}}
\def\SIGyr{\,\mathrm{Gyr}}
\def\SIV{\,\mathrm{V}}
\def\SImV{\,\mathrm{mV}}
\def\SIeV{\,\mathrm{eV}}
\def\SIkeV{\,\mathrm{keV}}
\def\SIMeV{\,\mathrm{MeV}}
\def\SIGeV{\,\mathrm{GeV}}
\def\SIcal{\,\mathrm{cal}}
\def\SIkcal{\,\mathrm{kcal}}
\def\SImol{\,\mathrm{mol}}
\def\SIN{\,\mathrm{N}}
\def\SIHz{\,\mathrm{Hz}}
\def\SIm{\,\mathrm{m}}
\def\SIcm{\,\mathrm{cm}}
\def\SIfm{\,\mathrm{fm}}
\def\SImm{\,\mathrm{mm}}
\def\SInm{\,\mathrm{nm}}
\def\SImum{\,\mathrm{\mu m}}
\def\SIJ{\,\mathrm{J}}
\def\SIW{\,\mathrm{W}}
\def\SIkJ{\,\mathrm{kJ}}
\def\SIs{\,\mathrm{s}}
\def\SIkg{\,\mathrm{kg}}
\def\SIg{\,\mathrm{g}}
\def\SIK{\,\mathrm{K}}
\def\SImmHg{\,\mathrm{mmHg}}
\def\SIPa{\,\mathrm{Pa}}


%\cpic{<尺寸>}{<文件名>}}用于生成居中的图片。
\newcommand{\cpic}[2]{
\begin{center}
\includegraphics[scale=#1]{#2}
\end{center}
}

%\cpicn{<尺寸>}{<文件名>}{<注释>}用于生成居中且带有注释的图片,其label为图片名。
\newcommand{\cpicn}[3]
{
\begin{figure}[h!]
\cpic{#1}{#2}
\caption{#3\label{#2}}
\end{figure}
}
\title{实验E3 材料真空兼容性测试和等离子特性研究}
\begin{document}
\maketitle  
	\begin{tabular}{|p{8em}|p{8em}|p{8em}|p{10em}|}
		\hline
		\large{实验方案} &\large{实验记录}  &\large{分析讨论} &\large{总成绩}\\
		\hline
		        &          &          &  \\
	    \hline
	\hline 
	年级、专业: &17级物理学 &组号:& 6 \\
	\hline
	姓名:& 徐昊霆 &学号:&17353071  \\
	\hline
	日期:& 2019.9.9 &教师签名: &  \\
    \hline	
        \end{tabular}
        
        1. 实验报告由三部分组成:
        
        1) 预习报告:(提前一周)认真研读实验讲义,弄清实验原理;实验所需的仪器设备、用具及其使用(强烈建议到实验室预习),完成讲义中的预习思考题;了解实验需要测量的物理量,并根据要求提前准备实验记录表格(由学生自己在实验前设计好,可以打印)。预习成绩低于50\%者不能做实验(实验D2和D3需要提前一周的周四完成预习报告交任课老师批改,批改通过后,才允许做实验)。
        
        2) 实验记录:认真、客观记录实验条件、实验过程中的现象以及数据。实验记录请用珠笔或者钢笔书写并签名(用铅笔记录的被认为无效)。保持原始记录,包括写错删除部分,如因误记需要修改记录,必须按规范修改。(不得输入电脑打印,但可扫描手记后打印扫描件);离开前请实验教师检查记录并签名。
        
    3) 分析讨论:处理实验原始数据(学习仪器使用类型的实验除外),对数据的可靠性和合理性进行分析;按规范呈现数据和结果(图、表),包括数据、图表按顺序编号及其引用;分析物理现象(含回答实验思考题,写出问题思考过程,必要时按规范引用数据);最后得出结论。
    实验报告就是预习报告、实验记录、和数据处理与分析合起来,加上本页封面。
    
    2. 每次完成实验后的一周内交实验报告。
    
    3. 实验报告建议双面打印。
\newpage
\tableofcontents
\section{实验原理与方案}

\subsection{实验目的}
\par 
1. 学习基本的真空知识和技术,掌握真空的获得和测量方法。
\par
2. 通过真空气体放电实验,验证帕邢定律。

3. 了解四极杆质谱仪工作原理,掌握质谱仪的操作,进行真空系统检漏和真空环境分
析。

4. 研究等离子体特性,获得等离子体基本参数。

5. 使用质谱仪测试不同材料在真空中的脱气性质,了解材料的真空兼容性。

6. 深入探讨四极质谱仪工作原理。
\subsection{仪器用具}
\begin{tabular}{c|c|c|c}
	\hline
        编号 & 仪器名称 &数量& 主要参数(型号,测量范围,精度) \\
	\hline 
	1 &  上海宜准公司  真空平台 &1 & VQP01\\
	\hline

\end{tabular}
\subsection{实验安全注意事项}

1. 操作前请检查真空腔体是否密封,检查高压电源开关、分子泵电源开关是否断开,以及应急按钮是否断开。

2. 注意高电压电源使用安全。(高压电源受真空计控制,实验前请确认真空计是否通电;通电情况下请勿插拔高压电源后面板高压输出接口,切勿接触后侧电力控制部分;实验前请检查高压电源调节旋钮,务必置零;实验过程中请勿接触高压电源后面板以及高压电源内侧结构。)

3. 若实验中用到分子泵,需机械泵先抽真空压强低于 10 pa 以下才能开启分子泵电源。

4. 若实验中用到四极质谱仪,开启四极质谱仪时保证真空压强低于 $5.0\times 10^{-2}\SIPa$。

\subsection{实验原理}

真空的获得和测量:在给定空间内,气体压强低于一个大气压的气体状态,称之为真空。真空的获得就是人们常说的“抽真空”,即利用各种真空泵将被抽容器中的气体抽出,使该空间的压强低于一个大气压。真空测量是指用特定的仪器和装置,对某一特定空间内真空高低的测定,这种仪器或装置称为真空计(仪器、规管)。

固体对气体的吸附及气体的脱附:气体吸附就是固体表面捕获气体分子的现象,吸附分为物理吸附和化学吸附。其中物理吸附没有选择性,任何气体在固体表面均可发生,主要靠分子间的相互吸引力引起的。物理吸附的气体容易发生脱附,而且这种吸附只在低温下有效;化学吸附则发生在较高的温度下,与化学反应相似,气体不易脱附,但只有当气体和固体表面原子接触生成化合物时才能产生吸附作用。气体的脱附是气体吸附的逆过程。通常把吸附在固体表面的气体分子从固体表面被释放出来的过程叫做气体的脱附。

气体放电、等离子体和帕邢定律:气体放电的基本过程是利用外(电)场加速电子使
之碰撞中性原子(分子)来电离气体。等离子体由离子、电子以及未电离的中性原子(分子)
的集合组成,整体呈中性的物质状态。气体放电是产生等离子体的一种常见形式。帕邢定律是表征均匀电场气体间隙击穿电压、间隙距离和气压间关系的定律。

四极质谱仪:四极杆上加有直流和射频交流分量电压(势),使得一定质量电荷比的离子可稳定的通过四极杆质量过滤器(离子能够稳定地通过四极电场),而不会撞上或逸出四极杆,可将离子根据质量电荷比进行过滤分类,归纳成质谱。
\newpage
\section{实验步骤与记录}
\begin{tabular}{|p{9em}|p{9em}|p{9em}|p{9em}|}
	\hline 
	专业:     &Physics       &年级:      & 17     \\
	\hline
	姓名:& 徐昊霆 &学号:&17353071  \\
	\hline
	室温:&                    &实验地点 & 教学楼 \\
	\hline	
	学生签名: & & 评分: & \\
	\hline
	日期: & 2019.9.9 & 教师签名:&  \\
	\hline
\end{tabular}
\subsection{实验记录}
\subsection{实验中遇到的问题记录}
\newpage
\section{分析与讨论}
\begin{tabular}{|p{9em}|p{9em}|p{9em}|p{9em}|}
	\hline 
	专业:     &Physics       &年级:      & 17     \\
	\hline
	姓名:& 徐昊霆、谢梓冰 &学号:&17353071、17353070  \\
	\hline
	日期&     2019.5.31               & &  \\
	\hline	
	评分 & & 教师签名 & \\
	\hline
\end{tabular}
\subsection{实验后思考题}

\end{document}
